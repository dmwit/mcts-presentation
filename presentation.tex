\documentclass[table]{beamer}
\usepackage[draft]{beamer_minimal}
\usepackage{yfonts}
\usepackage[normalem]{ulem}
\usepackage{tikz}

% https://tex.stackexchange.com/questions/89588/positioning-relative-to-page-in-tikz
\makeatletter
\def\parsecomma#1,#2\endparsecomma{\def\page@x{#1}\def\page@y{#2}}
\tikzdeclarecoordinatesystem{page}{
    \parsecomma#1\endparsecomma
    \pgfpointanchor{current page}{north east}
    % Save the upper right corner
    \pgf@xc=\pgf@x%
    \pgf@yc=\pgf@y%
    % save the lower left corner
    \pgfpointanchor{current page}{south west}
    \pgf@xb=\pgf@x%
    \pgf@yb=\pgf@y%
    % Transform to the correct placement
    \pgfmathparse{(\pgf@xc-\pgf@xb)/2.*\page@x+(\pgf@xc+\pgf@xb)/2.}
    \expandafter\pgf@x\expandafter=\pgfmathresult pt
    \pgfmathparse{(\pgf@yc-\pgf@yb)/2.*\page@y+(\pgf@yc+\pgf@yb)/2.}
    \expandafter\pgf@y\expandafter=\pgfmathresult pt
}
\makeatother

\colorlet{alertgreen}{green!55!black}
\colorlet{alertred}{red!70!black}
\setbeamercolor{alerted text}{fg=alertgreen}
\newcommand\img[2]{\includegraphics[width=#1\textwidth]{images/#2}}

\title{Win any game with these 3 weird tricks.}
\subtitle{Number three will shock you\\[2em]\img{0.5}{number-three-will-shock-you.jpg}}
\author{Daniel Wagner} % for hyperref
\date{September 27, 2019}

\begin{document}

\maketitle

\begin{frame}
	\frametitle{What's the plan, Stan?}
	\begin{itemize}
		\setlength\itemsep{5ex}
		\item \alert{Intro to minimax}
		\item What if minimax didn't suck?
		\item How to \sout{win} lose slowly in Vegas
		\item What if our priors didn't suck?
	\end{itemize}
\end{frame}

\begin{frame}
	\frametitle{Show of hands}
	\begin{center}
		\img{0.7}{hands-up.jpg}
	\end{center}
\end{frame}

\begin{frame}
	\frametitle{Games, generally}
	\begin{center}
		\img{1}{games-1.jpg}
	\end{center}
	% for alignment with next slide
	\begin{tikzpicture}
	\end{tikzpicture}
\end{frame}

\begin{frame}
	\frametitle{Games, generally}
	\begin{center}
		\img{1}{games-1.jpg}
	\end{center}
	\begin{tikzpicture}[remember picture,overlay]
		\node[draw,circle,alertgreen] at (page cs:-0.21,0.525) (node) {\phantom{blargle}};
		\node[alertgreen] at (page cs:0.2,0.7) (question) {Which move?};
		\path[draw,alertgreen,->] (question) -> (node);
	\end{tikzpicture}%
\end{frame}

\begin{frame}
	\frametitle{\begin{tabular}{m{7em}m{0.21\linewidth}}Intro to minimax&\img{0.2}{mini-max.png}\end{tabular}}
	\begin{itemize}
		\item Platonic ideal
		\item Ur-search
		\item Mathematician's first love
			\pause
			\begin{center}
				\img{0.15}{superman.jpg}
			\end{center}
			\pause
		\item Simple, elegant, obviously correct\pause\ldots\ and
			{\color{alertred}utterly useless}
	\end{itemize}
\end{frame}

\begin{frame}
	\frametitle{How to minimax}
	\begin{enumerate}
		\item Start at the leaves, and work towards the root.
		\item Each player picks the move that's best for them.
		\item ????
		\item Profit.
	\end{enumerate}

	\\[5ex]

	Outcome: a game tree where each node is labeled by the \alert{best
	worst-case} utility that each player can guarantee for themselves.
\end{frame}

\begin{frame}
	\frametitle{Minimax illustrated}
	\begin{center}
		\img{1}{games-1.jpg}
	\end{center}
\end{frame}

\begin{frame}
	\frametitle{Minimax illustrated}
	\begin{center}
		\img{1}{games-2.jpg}
	\end{center}
\end{frame}

\begin{frame}
	\frametitle{Minimax illustrated}
	\begin{center}
		\img{1}{games-3.jpg}
	\end{center}
\end{frame}

\begin{frame}
	\frametitle{Minimax illustrated}
	\begin{center}
		\img{1}{games-4.jpg}
	\end{center}
\end{frame}

\begin{frame}
	\frametitle{Minimax illustrated}
	\begin{center}
		\img{1}{games-5.jpg}
	\end{center}
\end{frame}

\begin{frame}
	\frametitle{Minimax illustrated}
	\begin{center}
		\img{1}{games-6.jpg}
	\end{center}
\end{frame}

\begin{frame}
	\frametitle{Minimax illustrated}
	\begin{center}
		\img{1}{games-7.jpg}
	\end{center}
\end{frame}

\begin{frame}
	\frametitle{What's the plan, Stan?}
	\begin{itemize}
		\setlength\itemsep{5ex}
		\item Intro to minimax
		\item \alert{What if minimax didn't suck?}
		\item How to \sout{win} lose slowly in Vegas
		\item What if our priors didn't suck?
	\end{itemize}
\end{frame}

\begin{frame}
	\frametitle{Why does it suck?}
	\begin{enumerate}
		\item Code up your minimax algorithm.
		\item Teach it the rules of chess.
		\item Set it off running.
	\end{enumerate}

	\begin{center}
		\img{0.4}{hourglass.jpg}
	\end{center}

	You slowly realize that the game tree has $>2^{64}$ leaves\ldots
\end{frame}

\begin{frame}
	\frametitle{Solution}
	\begin{enumerate}
		\item Only descend into the tree a little bit\\(not all the way to
			leaves).
		\item However deep we made it, \emph{guess}\footnotemark{} minimax's valuation.
		\item Minimax our way back to the root.
	\end{enumerate}
	\footnotetext{Completely computationally intractable.}
\end{frame}

\begin{frame}
	\frametitle{Weird Trick \#1: Rollouts}
	\alert{Sample} from a node's leaves to guess it's valuation.

	\\[8ex]

	(Typically: repeatedly choose random moves until the game ends.)
\end{frame}

\begin{frame}
	\frametitle{What's the plan, Stan?}
	\begin{itemize}
		\setlength\itemsep{5ex}
		\item Intro to minimax
		\item What if minimax didn't suck?
		\item \alert{How to \sout{win} lose slowly in Vegas}
		\item What if our priors didn't suck?
	\end{itemize}
\end{frame}

\begin{frame}
	\frametitle{How to \sout{win} lose slowly in Vegas}
	\begin{center}
		\img{0.4}{slot-machine-choice-1.png}\hfil\img{0.4}{slot-machine-choice-2.png}

		\\[5ex]

		There's a lot of slot machines.

		Are they all equally terrible?
	\end{center}
\end{frame}

\begin{frame}
	\frametitle{How to lose quickly}
	\begin{itemize}
		\item Sit down at a slot machine and play it nonstop.
			\begin{itemize}
				\item Maybe you picked the worst one.
				\item Probably you picked a merely mediocre one.
				\item You will only ever learn how terrible the one slot
					machine is.
			\end{itemize}
		\item Play the slot machines round-robin.
			\begin{itemize}
				\item Good risk management: only $1/n$ of your resources
					spent on worst machine.
				\item You get about equal confidence judgments of each
					machine.
				\item \ldots{}but you never take advantage of that
					knowledge.
			\end{itemize}
	\end{itemize}
\end{frame}

\begin{frame}
	\frametitle{Formal problem statement}
	\begin{itemize}
		\item You have a collection $D_1,\ldots,D_n$ of unknown distributions with
			support $[0,1]$.
		\item You repeatedly pick one of the distributions and sample from
			it, giving you a collection of samples $s_1,\ldots,s_m$.
		\item One of the distributions, say, distribution $D_t$, has the
			highest mean $\mu_t$.
		\item Your \alert{regret} is the expected difference
			\[m\mu_t - \sum_{i=1}^m s_i\]
			between the best expected payout and your payout.
	\end{itemize}

	What algorithm minimizes regret?
\end{frame}

\begin{frame}
	\frametitle{UCB1}
	Keep statistics $\bar p$ and $\bar c$ for each distribution:
	\begin{itemize}
		\item $p_1,\ldots,p_n$ is the \alert{payout} from each distribution (the sum
			of the samples)
		\item $c_1,\ldots,c_n$ is the \alert{count} of how many times you
			selected each distribution
	\end{itemize}
	Set $c=\sum_{j=1}^n c_j$. Select the distribution $D_i$ that maximizes:
	\[\frac{p_i}{c_i} + \sqrt{\frac{2\ln(c)}{c_i}}\]

	\pause

	\begin{center}
		\img{0.4}{confused.jpg}
	\end{center}
\end{frame}

\begin{frame}
	\frametitle{Miscellanea about UCB1}

	\begin{itemize}
		\item Second term is a one-sided confidence interval.
		\item Chooses the best distribution exponentially more often than
			any other in the long run.
		\item $O(\ln(m))$ regret after $m$ selections.
		\item Asymptotically optimal regret.
	\end{itemize}
\end{frame}

\begin{frame}
	\frametitle{Weird Trick \#2: UCT1}
	I have time to search another node in the game tree.

	Which one should I search?

	\\[3ex]

	\begin{enumerate}
		\item Start at the root.
		\item \alert{Use UCB1 to decide which child to descend to.}
		\item Repeat until we find a node we haven't searched yet.
	\end{enumerate}
\end{frame}

\begin{frame}
	\frametitle{What's the plan, Stan?}
	\begin{itemize}
		\setlength\itemsep{5ex}
		\item Intro to minimax
		\item What if minimax didn't suck?
		\item How to \sout{win} lose slowly in Vegas
		\item \alert{What if our priors didn't suck?}
	\end{itemize}
\end{frame}

\begin{frame}
	\frametitle{Obvious positions}
	\begin{center}
		\only<+>{\img{0.4}{go-position-1.png}}
		\only<+>{\img{0.4}{go-position-2.png}}

		\\[3ex]

		\onslide<.>
		\img{0.2}{satisfied.jpg}
	\end{center}
\end{frame}

\begin{frame}
	\frametitle{Obvious moves: go}
	\begin{center}
		\only<+>{\img{0.4}{go-moves-1.png}}
		\only<+>{\img{0.4}{go-moves-2.png}}

		\\[3ex]

		\onslide<.> 26 legal moves
	\end{center}
\end{frame}

\begin{frame}
	\frametitle{Obvious moves: chess}
	\begin{center}
		\only<+>{\img{0.4}{chess-moves-1.png}}
		\only<+>{\img{0.4}{chess-moves-2.png}}

		\\[3ex]

		\onslide<.> 24 legal moves
	\end{center}
\end{frame}

\begin{frame}
	\frametitle{Meanwhile, in MCTS land\ldots}
	\begin{itemize}
		\item Start with very low-quality evaluation
		\item Rollouts select moves uniformly
		\item So any single rollout gives very low-quality information
		\item Convergence is very slow!
	\end{itemize}

	\\[5ex]

	What if we spent a bit more time and did slightly better?
\end{frame}

\begin{frame}
	\frametitle{Weird Trick \#3: Deep Learning}
	Use a \alert{\textgoth{Deep Convolutional Neural Net}}\footnotemark{} for
	
	\\[1.5ex]

	\begin{itemize}
		\setlength\itemsep{1.5ex}
		\item initial evaluation estimates, and
		\item more informative rollouts.
	\end{itemize}

	\\[1.5ex]

	Train it on games played by pure MCTS.

	\footnotetext{Okay, so there was a ``dunno, wave your hands and throw ML
	at it'' moment after all.}
\end{frame}

\begin{frame}
	\frametitle{What's the plan, Stan?}
	\begin{itemize}
		\setlength\itemsep{5ex}
		\item Intro to minimax
		\item What if minimax didn't suck?
		\item How to \sout{win} lose slowly in Vegas
		\item What if our priors didn't suck?
	\end{itemize}
\end{frame}

\begin{frame}
	\frametitle{THE END}
	\begin{center}
		\img{0.8}{questions.jpg}
	\end{center}
\end{frame}

\begin{frame}
	\frametitle{Objections}
	\begin{center}
		\img{0.35}{trap-card.jpg}
	\end{center}

	\begin{itemize}
		\item You baited me with ``maximize worst-case'', but switched to
			``maximize expected case''.
			\begin{itemize}
				\item Turns out to be the same, in the limit!

					\ldots{}but nobody cares, because the limit is really
					far away.
				\item Better priors help.
			\end{itemize}
		\item UCB1 works when the set of distributions is fixed. But we keep
			adding distributions as we search.
			\begin{itemize}
				\item True. A separate proof shows it works anyway.
			\end{itemize}
		\item If all we have is a tree, what the heck are we convolving?
			\begin{itemize}
				\item That part is specific to games that have a board and
					pieces. (That's most games humans play.)
			\end{itemize}
	\end{itemize}
	\addtocounter{framenumber}{-1}
\end{frame}

\end{document}
